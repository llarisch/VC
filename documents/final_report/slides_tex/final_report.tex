\documentclass[11pt]{beamer}
\usepackage{amsmath}
\usepackage{amssymb}
\usepackage{amsthm}
\usepackage[utf8]{inputenc}
%\usepackage[margin=0.5in]{geometry}
\usepackage{graphicx}
\usepackage{algorithm}
\usepackage{algpseudocode}
\usepackage{xcolor,colortbl}
\usepackage{url}
\usepackage{caption}
%\usepackage{algorithmic}
\usepackage{algorithm}
\usepackage{comment}

\newcommand{\mc}[2]{\multicolumn{#1}{c}{#2}}
\definecolor{Gray}{gray}{0.85}

\newcolumntype{a}{>{\columncolor{Gray}}c}
\newcolumntype{b}{>{\columncolor{white}}c}


\newtheorem*{proposition}{Proposition}
\newtheorem*{proof_}{proof}

\DeclareMathOperator{\tw}{tw}
\DeclareMathOperator{\VC}{VC}
\DeclareMathOperator{\CLIQUE}{CLIQUE}
\DeclareMathOperator{\reach}{reach}

\title{final report for CS260: \\ \normalsize the vertex cover problem}
\author{Osayd Abdu (142461), Abdulelah Alneghaimish (159296), \\ Lukas Larisch (154273), Elaf Islam (142724)}
\date{}

\parindent 0pt


\begin{document}

\begin{frame}

\maketitle

\end{frame}

\begin{frame}
\frametitle{Motivation from theory}

\begin{theorem}[Courcelle, 1990]
Every graph property definable in monadic second-order logic can be decided in linear time on graphs of bounded treewidth.
\end{theorem}

\begin{itemize}
\item most problems from NP are definable in MSO
\item [] $\rightarrow$ max-CLIQUE
\item [] $\rightarrow$ min-VERTEX COVER
\item no experimental evaluation of this theorem yet
\end{itemize}

\end{frame}




\begin{frame}
\frametitle{Recap: Vertex Cover}

$C \subseteq V(G)$, such that $\{u, v\} \in E(G) \implies u \in C \lor v \in C$. \\

\begin{center}

{
\centering
\includegraphics[scale=0.5]{example_cover}
}

\end{center}

\end{frame}


\begin{frame}
\frametitle{Recap: Clique}

$C \subseteq V(G)$, such that $u, v \in C \implies \{u, v\} \in E(G)$. \\


\begin{center}

{
\centering
\includegraphics[scale=0.2]{example_clique}
}

\end{center}

\end{frame}


\begin{frame}
\frametitle{Recap: they are poly-time reducable}

$(G, k) \in \VC \iff (\bar G, |V(G)|-k) \in \CLIQUE$
\end{frame}


\begin{frame}
\frametitle{Recap: First algorithm}

\begin{itemize}
\item exponential time
\item Basically:
\begin{itemize}
\item For $v$, we investigate $v \in \CLIQUE$ and $v \not \in \CLIQUE$.
\item $v \in \CLIQUE$: Remove all vertices not connected to $v$ and recurse.
\item $v \not \in \CLIQUE$: Remove $v$ and recurse.
\vspace*{5mm}
\item [] $\rightarrow$ $\mathcal{O}(2^n \cdot \text{poly}(n))$
\end{itemize}
\end{itemize}

\end{frame}

\begin{frame}
\frametitle{Recap: Second algorithm}

\begin{itemize}
\item exponential time for unbounded treewidth
\item otherwise, $\mathcal{O}(2^{tw(G)+1} \cdot n^{O(1)}) = \mathcal{O}(n)$.

\item Compute a tree decomposition of minimum size for the input graph
\item Convert it to a nice tree decomposition (can be handled better in later steps)
\item Dynamic Programming on the nice tree decomposition (extension of approach on trees, see HW3 for Independent Set)
\end{itemize}

\end{frame}


\begin{frame}
\frametitle{Graphs}

\begin{itemize}
\item Control flow graphs
\item random partial $k$-trees 
\begin{itemize}
\item $k \in \{1, \dots, 15\}$
\item $n \in \{50,100,200,250,500\}$
\item $p \in \{.97, .95, .90, .80, .70\}$ 
\item 5 each
\end{itemize}
\item "named" graphs, e.g.
\begin{itemize}
\item Petersen 
\item world map graph (countries adjacent  $\rightarrow$ edge)
\end{itemize}
\end{itemize}
\end{frame}



\begin{frame}
\frametitle{Statistics: Control flow graphs}

\begin{center}
\footnotesize
\begin{table}[h!]
\centering
\begin{tabular}{|c|c|c|c|c|c|}
\hline
\#graphs & avg/med./max \#vert. & avg/med./max \#edges & avg/med./max $\tw$ \\
\hline \hline
142 & 37.47/23/544 & 39.45/24/609 & 1.85/2/4 \\
\hline
1082 & 36.49/19/1452 & 38.08/19/1591 & 1.71/2/7 \\
\hline
529 & 42.07/24/587 & 45.13/24/668 & 1.97/2/6 \\
\hline
\end{tabular}
\captionof{table}{Statistics for the considered CFGs.}
\label{stat_CFGs}
\end{table}
\end{center}

\end{frame}


\begin{frame}
\frametitle{Statistics: random partial $k$-trees (1)}

\begin{center}
\begin{table}[h!]
\centering
\begin{tabular}{|c|c|c|c|}
\hline
$n$ & avg/med./max \#edges & avg/med./max $\tw$ \\
\hline \hline
50 & 48.11/30/218 & 3.66/2/15 \\
\hline
100 & 102.11/61/441 & 4.46/3/15 \\
\hline
200 & 211.70/90/908 & 5.07/3/15 \\
\hline
250 & 264.48/153/1116 & 5.27/4/15 \\
\hline
500 & 537.28/310/2236 & 5.95/5/15 \\
\hline
\end{tabular}
\captionof{table}{Statistics for the considered partial $k$-trees grouped by $n$.}
\end{table}
\end{center}

\end{frame}


\begin{frame}
\frametitle{Statistics: random partial $k$-trees (2)}

\begin{center}
\small
\begin{table}[h!]
\centering
\begin{tabular}{|c|c|c|c|}
\hline
$k$ & avg/med./max \#edges & avg/med./max $\tw$ \\
\hline \hline
1 & 29.95/16/165 & 0.97/1/1 \\
\hline
2 & 60.70/30/329 & 1.40/1/2 \\
\hline
3 & 88.3/47/485 & 1.75/2/3 \\
\hline
4 & 118.58/62/626 & 2.30/2/4 \\
\hline
5 & 147.68/77/787 & 2.76/2/5 \\
\hline
6 & 178.70/94/936 & 3.43/3/6 \\
\hline
7 & 204.95/106/1082 & 4.09/4/7 \\
\hline
8 & 234.02/123/1189 & 4.79/5/8 \\
\hline
9 & 262.10/143/1370 & 5.31/6/9 \\
\hline
10 & 290.24/149/1492 & 6.06/7/10 \\
\hline
11 & 318.78/166/1658 & 6.70/8/11 \\
\hline
12 & 348.72/169/1836 & 7.46/8/12 \\
\hline
13 & 375.87/187/1976 & 8.04/9/13 \\
\hline
14 & 404.34/210/2050 & 8.82/10/14 \\
\hline
15 & 431.58/218/2236 & 9.45/11/15 \\
\hline
\end{tabular}
\captionof{table}{Statistics for the considered $k$-trees grouped by $k$.}
\end{table}
\end{center}

\end{frame}

\begin{frame}
\frametitle{Statistics: "named" graphs}

\begin{center}
\footnotesize
\begin{table}[h!]
\centering
\begin{tabular}{|c|c|c|c|c|}
\hline
\#graphs & avg/med./max \#vert. & avg/med./max \#edges & avg/med./max $\tw$ \\
\hline \hline
125 & 83.05/25/3282 & 166.63/60/6561 & 9.79/7/29 \\
\hline
\end{tabular}
\captionof{table}{Statistics for the named graphs.}
\end{table}
\end{center}

\begin{center}
\begin{figure}[h!]
\includegraphics[scale=0.06]{Petersen}
\hspace*{3mm}
\includegraphics[scale=0.06]{Balaban}
\hspace*{3mm}
\includegraphics[scale=0.12]{Ljubljana}
\hspace*{3mm}
\includegraphics[scale=0.06]{Foster}
\caption{The Petersen, Balaban 11-cage, Ljubljana and Foster graphs.}
\end{figure}
\end{center}

Treewidths: treewidth 4, treewidth 25, treewidth 25, treewidth 22.

Last three are hardest for experiments.

\end{frame}


\begin{frame}
\frametitle{Experiments}

\begin{itemize}
\item C++
\item boost graphs
\item TdLib for the provision of tree decompositions in algo 2 (won PACE 2017)
\item no graphs of treewidth $>$ 30
\end{itemize}
\end{frame}



\begin{frame}
\frametitle{Results: Control flow graphs}

\begin{center}
\footnotesize
\begin{table}[h!]
\centering
\begin{tabular}{|c|c|c|c|c|c|}
\hline
\#graphs & avg/med./max $\tw$. & avg/med./max $\VC$ & avg/med./max time[s] \\
\hline \hline
142 & 1.85/2/4 & 18.56/11/269 & 0.0008/0.0003/0.0307 \\
\hline
1082 & 1.71/2/7 & 18.07/9/728 & 0.0007/0.0002/0.1559 \\
\hline
529 & 1.97/2/6 & 20.76/12/295 & 0.0008/0.0003/0.0339 \\
\hline
\end{tabular}
\captionof{table}{Results for the considered CFGs.}
\end{table}
\end{center}

\begin{itemize}
\item all graphs in less than 1 second
\item low treewidth (bounded by 7 + \#gotos)
\end{itemize}

\end{frame}



\begin{frame}
\frametitle{Results: random partial $k$-trees (1)}

\begin{center}
\footnotesize
\begin{table}[h!]
\centering
\begin{tabular}{|c|c|c|c|c|}
\hline
$n$ & avg/med./max $\tw$. & avg/med./max $\VC$ & avg/med./max time[s] \\
\hline \hline
50 & 3.66/2/15 & 12.23/12/27 & 0.1347/0.0012/2.6110 \\
\hline
100 & 4.46/3/15 & 20.35/20/44 & 0.2666/0.0019/11.5305 \\
\hline
200 & 5.07/3/15 & 34.99/33/69 & 0.7391/0.0033/23.1289 \\
\hline
250 & 5.31/4/15 & 42.45/39/88 & 0.9472/0.0103/24.0645 \\
\hline
500 & 5.95/5/15 & 78.49/72/157 & 2.2913/0.0298/43.4646 \\
\hline
\end{tabular}
\captionof{table}{Results for the considered partial $k$-trees grouped by $n$.}
\end{table}
\end{center}

Running time linear in $n$.

\end{frame}


\begin{frame}
\frametitle{Results: random partial $k$-trees (2)}

\begin{center}
\footnotesize
\begin{table}[h!]
\centering
\begin{tabular}{|c|c|c|c|c|}
\hline
package & avg/med./max $\tw$. & avg/med./max $\VC$ & avg/med./max time[s] \\
\hline \hline
1 & 0.97/1/1 & 18.33/11/86 & 0.0042/0.0029/0.0130 \\
\hline
2 & 1.4/1/2 & 26.07/19/117 & 0.0036/0.0022/0.0127 \\
\hline
3 & 1.75/2/3 & 29.4/19/118 & 0.0043/0.0028/0.0195 \\
\hline
4 & 2.3/2/4 & 33.05/23/129 & 0.0064/0.0031/0.0309 \\
\hline
5 & 2.76/2/5 & 34.94/24/134 & 0.0096/0.0038/0.0525 \\
\hline
6 & 3.43/3/6 & 37.14/27/143 & 0.0171/0.0045/0.1015 \\
\hline
7 & 4.09/4/7 & 38.28/29/146 & 0.0308/0.0081/0.1774 \\
\hline
8 & 4.79/5/8 & 39.62/30/144 & 0.0542/0.0116/0.4428 \\
\hline
9 & 5.31/6/9 & 41.00/32/144 & 0.1058/0.0150/0.9290 \\
\hline
10 & 6.06/7/10 & 42.64/32/148 & 0.1780/0.0280/1.3136 \\
\hline
11 & 6.70/8/11 & 42.45/39/88 & 0.3626/0.0272/2.8060 \\
\hline
12 & 7.46/8/12 & 44.33/35/147 & 0.7838/0.03882/8.2476 \\
\hline
13 & 8.04/9/13 & 44.92/35/154 & 1.5113/0.0586/12.1212 \\
\hline
14 & 8.82/10/14 & 45.42/36/146 & 3.7556/0.1338/42.6301 \\
\hline
15 & 9.45/11/15 & 47.46/38/157 & 6.2944/0.2169/44.1330 \\
\hline
\end{tabular}
\captionof{table}{Results for the considered partial $k$-trees.}
\end{table}
\end{center}

\end{frame}


\begin{frame}
\frametitle{Results: "named" graphs}

TODO

\end{frame}


\begin{frame}
\frametitle{Conclusions}

We experimentally showed that TD-based algorithms may be of practical use for solving NP-complete problems, here the vertex cover problem, on graph classes of bounded treewidth by comparing a standard exponential-time algorithm with an TD-based dynamic programming algorithm on control flow graphs derived from various C functions and random partial $k$-trees for small $k$, both having small treewidth. Furthermore we tested both algorithms on famous graphs, i.e. having a name; this test set contains graphs of prohibitively high treewidth, such that some graph have to be excluded for our experiments. Nevertheless, an optimal solution for the vertex cover problem could be computed in reasonable time by the TD-based algorithm for graphs having treewidth up to 29. The exponential time algorithm performs poorly on most graphs, because it does not make use of the special structure of graphs having low treewidth. Neglecting structural properties of graphs arising from treewidth, it exponentially depends on the size of input graphs, hence having no chance to deal graphs with a "large" number of vertices.

\end{frame}



\end{document}